%! Author = Drackaro
%! Date = 9/1/2024

% Preamble
\documentclass[11pt]{article}

%Packages
\usepackage{amsmath}
\usepackage{graphicx}
\usepackage{amssymb}
\usepackage{color}
\usepackage{listings}

\author{
    Greedy
    }
\title{Problema \#1: Scooter}

% Document
\begin{document}


    \maketitle
    \newpage

    \section{Problemática}
    \subsection{¿De qué trata el problema?}
    El campus de una universidad cuenta con n edificios, numerados desde 1 hasta n. En cada uno de estos
    edificios puede haber planeada una clase de matemáticas, o una clase de programación, o ninguna clase
    (nunca hay planeadas clases de ambas materias en un mismo edificio). Además de esto, en cada edificio
    hay a lo sumo un profesor, y cada profesor es experto de una de las materias, es decir, que hay profesores
    de matemática y de programación.
    
    Como trabajador de University Express Inc., tu trabajo es transportar de manera rápida y amena  a los
    profesores para que puedan impartir sus clases. Para esto se te ha otorgado nada más y nada menos que un
    scooter (una motorina vamos), en la que cabes tú y a lo sumo un pasajero.
    
    Inicialmente serás la única persona en el scooter. Cuando llegues a uno de los edificios de la universidad
    puedes dejar o recoger a un profesor en dicho edificio. Para conseguir tu tarea se te ha permitido conducir
    a cada uno de los n edificios a lo sumo una vez, en el orden que desees (tambien puedes elegir en qué edificio
    empezar)

    Al final de tu recorrido, en cada edificio donde haya planeada una clase de matemáticas debe haber un
    profesor experto en dicha materia, mientras que en cada edificio con una clase de programación planeada
    debe haber un profesor experto en esta materia. Planea un itinerario de viajes con el que puedan ser
    impartidas con éxito todas las clases planeadas.

    \subsection{Entrada}
    La entrada del problema constará de 3 elementos:

    \begin{itemize}
        \item Un número entero  n $(1 \leq n \leq 2000)$ que será la cantidad de edificios de la universidad
        \item Un string de longitud n conformado por los caracteres (\texttt{M}, \texttt{P}, \texttt{-}), el caracter
              en la posición i determina la materia de la clase que está programada en el edificio i: \texttt{M} representa
              una clase de matemáticas, \texttt{P} una de programación y \texttt{-} representa que no hay ninguna
              clase planeada en ese edificio.
        \item Un string de longitud n conformado otra vez por los caracteres (\texttt{M}, \texttt{P}, \texttt{-}),
              pero esta vez para definir los profesores que se encuentran inicialmente en cada edificio. Una \texttt{M}
              o \texttt{P} en la posición i representa que en el edificio i hay un profesor de matemáticas o programación
              respectivamente mientras una \texttt{-} representa que dicho edificio está vacío.
    \end{itemize}
    
\end{document}