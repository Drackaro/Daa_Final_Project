%! Author = Drackaro
%! Date = 9/1/2024

% Preamble
\documentclass[11pt]{article}

%Packages
\usepackage{amsmath}
\usepackage{graphicx}
\usepackage{amssymb}
\usepackage{color}
\usepackage{listings}

\author{
    DP
    }
\title{Problema \#2: Fixed Points}

% Document
\begin{document}


    \maketitle
    \newpage

    \section{Problemática}
    \subsection{¿De qué trata el problema?}
    Considera una secuencia de números enteros $a_1, a_2, ... , a_n$. En un movimientopuedes seleccionar 
    cualquier elemento de la secuencia y eliminarlo. Luego de que un elemento se elimine todos los demás 
    elementos a su derecha son desplazados a la izquierda una posición, por lo que no quedan elementos vacíos 
    en la secuencia. Luego de hacer un movimiento la longitud de la secuencia didminuye en 1. Los índices 
    de los elementos se recalculan luego de esto.

    Dada la secuencia $a_1, a_2, ... , a_n$ y un número k, debes encontrar la menor cantidad de movimientos 
    que se deben hacer para que la secuencia resultante tenga al menos k elementos que sean iguales a sus 
    índices, es decir, que la secuencia resultante $b_1, b_2, ... , b_m$ tenga al menos k índices i tal que 
    $b_i = i$

    \subsection{Entrada}
    Se reciben dos enteros n (longitud de la cadena) y k (número de elementos que deben coincidir con su índice),
    asi como la secuencia $a_1, a_2, ... , a_n$ $(1 \leq a_i \leq n)$. Los números de la secuencia no tienen 
    que ser diferentes necesariamente.
    
    \subsection{Salida}
    Se debe devolver un entero x $(0 \leq x \leq n)$, que represente la cantidad mínima de movimientos que se
    pueden hacer para que en la secuencia resultante existan al menos k elementos iguales a su índice. Si no
    es posible encontrar dicha secuencia se devuelve -1.
    
\end{document}