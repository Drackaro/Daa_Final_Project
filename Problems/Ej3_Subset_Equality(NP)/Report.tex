%! Author = Drackaro
%! Date = 9/1/2024

% Preamble
\documentclass[11pt]{article}

%Packages
\usepackage{amsmath}
\usepackage{graphicx}
\usepackage{amssymb}
\usepackage{color}
\usepackage{listings}

\author{
    NP Hard
    }
\title{Problema \#3: Subset Equality Problem}

% Document
\begin{document}

    \maketitle
    \newpage

    \section{Problemática}
    \subsection{¿De qué trata el problema?}
    Dado un conjunto S de números enteros no negativos, el problema consiste en identificar si existe una
    partición del conjunto S en dos nuevos conjuntos X y Y, de tal forma que la suma de todos los elementos
    que pertenecen a X sea igual a la suma de todos los elementos que pertenecen a Y.
    
    \subsection{Entrada}
    Un array S de números enteros no negativos.
    
    \subsection{Salida}
    True o False si es posible encontrar las particiones que cumplan la condición expuesta anteriormente, y en
    caso positivo devolver tambien dichas particiones. Si existe más de un par de particiones válidas se puede 
    devolver cualquiera.
    
    
\end{document}