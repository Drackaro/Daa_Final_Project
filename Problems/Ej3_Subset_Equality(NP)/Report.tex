%! Author = Drackaro
%! Date = 9/1/2024

% Preamble
\documentclass[11pt]{article}

%Packages
\usepackage{amsmath}
\usepackage{graphicx}
\usepackage{amssymb}
\usepackage{color}
\usepackage{listings}
\usepackage{hyperref}
\usepackage{array}

\title{Problema \#3: Subset Equality Problem}

\hypersetup{
    colorlinks=true,
    linkcolor=blue
}

% Document
\begin{document}

    \maketitle
    \newpage

    \tableofcontents
    \newpage

    \section{Problemática}
    \subsection{¿De qué trata el problema?}
    Dado un conjunto S de números enteros no negativos, el problema consiste en identificar si existe una
    partición del conjunto S en dos nuevos conjuntos X y Y, de tal forma que la suma de todos los elementos
    que pertenecen a X sea igual a la suma de todos los elementos que pertenecen a Y.

    De manera formal el problema se define de la siguiente manera: Sea $S = {s_1, s_2, ... ,s_n}$ se deben
    encontrar $X = x_1, x_2, ... , x_a$ y $Y = y_1, y_2, ... ,y_b$ tal que $a+b=n$, y:

    \[
    X \cup Y = S, X \cap Y= \emptyset, \sum_{i=0}^{a} x_i = \sum_{j=0}^b y_j
    \]
    
    \subsection{Entrada}
    Un array S de números enteros no negativos.
    
    \subsection{Salida}
    True o False si es posible encontrar las particiones que cumplan la condición expuesta anteriormente, y en
    caso positivo devolver tambien dichas particiones. Si existe más de un par de particiones válidas se puede 
    devolver cualquiera.

    \section{Demostración de NP Completitud}
    Para demostrar que un problema es NP completo debemos demostrar que dicho problema es tanto NP, como NP hard.
    Vamos a hacer esas demostraciones a continuación.

    \subsection{NP}
    Un problema es NP si, dada una solución candidata, es posible verificar en tiempo polinomial si la solución
    es correcta o no. En este caso una solución candidata a este problema está formada por los conjuntos de enteros
    $X$ y $Y$. Para verificar si la solución es correcta primero se debe comprobar si $X \cup Y = S$ y si
    $X \cap Y = \emptyset$, dicha comprobación puede hacerse en $O(n)$ simplemente comprobando si no hay elementos
    repetidos en los conjuntos y en estos están presentes todos los elementos de $S$. Aquí aclarar que los elementos
    se están comparando por referencia y no por valor, es decir, los elementos de $X$ y $Y$ no se comparan por sus 
    valores numéricos sino por su referencia en $S$, por ejemplo, si hay un 5 en $X$ y otro en $Y$ entonces 
    se tiene que cumplir que hay dos elementos 5 en $S$. Luego se verifica que $\sum_{i=0}^{a} x_i = \sum_{j=0}^b y_j$ 
    lo cual puede comprobarse haciendo un recorrido lineal por ambos conjuntos lo que tiene una complejidad de 
    $O(a+b) = O(n)$.

    Por tanto es posible verificar la correctitud de una solución candidata en tiempo polinomial ($O(n)$), por
    lo que queda demostrado que el problema pertenece al conjunto de los problemas NP.

    \subsection{NP Hard}
    Subset Equality problem es un caso particular de otro problema llamado Subset sum problem. En este se recibe un
    conjunto de entrada $S$ y un entero $k$, y lo que hay que buscar es un subconjunto de $S$ que cumpla que la suma 
    de todos sus elementos es exactamente $k$. Es fácil ver que Subset Equality problem en $S$ es quivalente a resolver 
    Subset sum con $S$ y $k = N \div 2$ donde $N = \sum_{i=0}^{n}x_i$ donde $x_i \in S$ para todo $i$. Por tanto lo que
    vamos a demostrar es que Subset sum es NP hard.
    
    Para demostrar que Subset Sum es NP hard vamos a realizar una reducción de un problema conocido que sabemos
    que es NP hard a este. El problema que vamos a utilizar es \textbf{3-Sat}, en este problema se tiene una expresión
    booleana en forma normal conjuntiva y cada cláusula de esta tiene 3 variables, se debe encontrar una asignación 
    a las variables (1 o 0) que haga verdadeda la expresión.

    Sea una expresión booleana con $n$ variables $x_i$ y $m$ cláusulas $c_j$. Por cada variable $x_i$ vamos
    a construir los números $t_i$ y $f_i$ cada uno de $n+m$ dígitos de la siguiente manera:

    \begin{itemize}
        \item El $i$-ésimo dígito de $t_i$ y $f_i$ es 1.
        \item Para $j$ con $n+1 \leq j \leq n+m$, el $j$-ésimo dígito de $t_i$ es 1 si la variable $x_i$ está
              en la clásula $c_{j-n}$.
        \item Para $j$ con $n+1 \leq j \leq n+m$, el $j$-ésimo dígito de $f_i$ es 1 si la variable $\neg x_i$ está
              en la clásula $c_{j-n}$.
        \item El resto de dígitos de $t_i$ y $f_i$ son 0.
    \end{itemize}

    \newpage

    Por ejemplo, para la siguiente expresión Booleana:

    \[
    (x_1 \vee x_2 \vee x_3) \wedge (\neg x_1 \vee \neg x_2 \vee x_3) \wedge (\neg x_1 \vee x_2 \vee \neg x_3)
    \wedge (x_1 \vee \neg x_2 \vee x_3)
    \]

    Los $t_i$ y $f_i$ quedan de la siguiente manera:


    \begin{table}[!h]
        \centering
        \begin{tabular}{|c|c|c|c|c|c|c|c|c|}
        \hline
        \multicolumn{1}{|c|}{} & \multicolumn{3}{c|}{\textbf{i}} & \multicolumn{4}{c|}{\textbf{j}} \\ \hline
        \textbf{Número} & 1 & 2 & 3 & 1 & 2 & 3 & 4 \\ \hline
        $t_1$ & 1 & 0 & 0 & 1 & 0 & 0 & 1 \\ \hline
        $f_1$ & 1 & 0 & 0 & 0 & 1 & 1 & 0 \\ \hline
        $t_2$ & 0 & 1 & 0 & 1 & 0 & 1 & 0 \\ \hline
        $f_2$ & 0 & 1 & 0 & 0 & 1 & 0 & 1 \\ \hline
        $t_3$ & 0 & 0 & 1 & 1 & 1 & 0 & 1 \\ \hline
        $f_3$ & 0 & 0 & 1 & 0 & 0 & 1 & 0 \\ \hline
        \end{tabular}
    \end{table}

    Ahora para cada cláusula $c_j$ vamos a construir los números $x_i$ y $y_i$ también de $n+m$ dígitos, de forma que
    el único dígito igual a 1 en las dos variables será el ()$n+j$)-ésimo, el resto serán 0. Siguiendo el ejemplo
    de la expresión booleana anterior los números que se forman son los siguientes:

    \begin{table}[!h]
        \centering
        \begin{tabular}{|c|c|c|c|c|c|c|c|c|}
        \hline
        \multicolumn{1}{|c|}{} & \multicolumn{3}{c|}{\textbf{i}} & \multicolumn{4}{c|}{\textbf{j}} \\ \hline
        \textbf{Número} & 1 & 2 & 3 & 1 & 2 & 3 & 4 \\ \hline
        $x_1$ & 0 & 0 & 0 & 1 & 0 & 0 & 0 \\ \hline
        $y_1$ & 0 & 0 & 0 & 1 & 0 & 0 & 0 \\ \hline
        $x_2$ & 0 & 0 & 0 & 0 & 1 & 0 & 0 \\ \hline
        $y_2$ & 0 & 0 & 0 & 0 & 1 & 0 & 0 \\ \hline
        $x_3$ & 0 & 0 & 0 & 0 & 0 & 1 & 0 \\ \hline
        $y_3$ & 0 & 0 & 0 & 0 & 0 & 1 & 0 \\ \hline
        $x_4$ & 0 & 0 & 0 & 0 & 0 & 0 & 1 \\ \hline
        $y_4$ & 0 & 0 & 0 & 0 & 0 & 0 & 1 \\ \hline
        \end{tabular}
    \end{table}

    Finalmente vamos a crear un último número $s$ de también $n+m$ dígitos de forma que los $n$ primeros números
    son 1, y el resto de $m$ números son 3. 
    
    Definamos una instacia de Subset sum donde el conjunto de entrada está formado por todos los números que hemos 
    construido excepto $s$, quien será el parámetro $k$, es decir, la suma requerida. Vamos a demostrar que si se 
    tiene una asignación de las variables que satisface la expresión (o sea se resuelve el 3-SAT), entonces existe un 
    subconjunto de números que satisface la instancia de subset sum que se creó.

    Teniendo la asignación que resuelve el 3-SAT vamos a armar el subconjunto de la siguiente forma:

    \begin{itemize}
        \item Tomamos $t_i$ si la variable $x_i$ es \texttt{True}
        \item Tomamos $f_i$ si la variable $x_i$ es \texttt{False}
        \item Tomamos $x_j$ si la cláusula $c_j$ tiene a lo sumo 2 variables \texttt{True}
        \item Tomamos $y_j$ si la cláusula $c_j$ tiene exactamente 1 variable \texttt{True}
    \end{itemize}

    Según el ejemplo que hemos estado siguiendo los números escogidos si consideramos que $x_1$, $x_2$ y $x_3$
    son \texttt{True} son los siguientes:

    \begin{table}[!h]
        \centering
        \begin{tabular}{|c|c|c|c|c|c|c|c|c|}
        \hline
        \multicolumn{1}{|c|}{} & \multicolumn{3}{c|}{\textbf{i}} & \multicolumn{4}{c|}{\textbf{j}} \\ \hline
        \textbf{Número} & 1 & 2 & 3 & 1 & 2 & 3 & 4 \\ \hline
        $t_1$ & 1 & 0 & 0 & 1 & 0 & 0 & 1 \\ \hline
        $t_2$ & 0 & 1 & 0 & 1 & 0 & 1 & 0 \\ \hline
        $t_3$ & 0 & 0 & 1 & 1 & 1 & 0 & 1 \\ \hline
        $x_2$ & 0 & 0 & 0 & 0 & 1 & 0 & 0 \\ \hline
        $y_2$ & 0 & 0 & 0 & 0 & 1 & 0 & 0 \\ \hline
        $x_3$ & 0 & 0 & 0 & 0 & 0 & 1 & 0 \\ \hline
        $y_3$ & 0 & 0 & 0 & 0 & 0 & 1 & 0 \\ \hline
        $x_4$ & 0 & 0 & 0 & 0 & 0 & 0 & 1 \\ \hline
        $s$   & 1 & 1 & 1 & 3 & 3 & 3 & 3 \\ \hline
        \end{tabular}
    \end{table}

    Vamos a demostrar por qué la suma de los elementos del conjunto siempre es igual a $s$ si lo construimos
    de esa manera:

    Primeramente vamos a analizar los números construidos de forma tabular como hemos hecho hasta ahora, de modo
    que para resolver el Subset sum las primeras $n$ columnas debem sumar 1 y el resto de las $m$ columnas debe sumar
    3. La primera parte se cumple ya que para cada $i$ se escoge a $t_i$ o a $f_i$, nunca a ambos ya que la variable
    $x_i$ solo tiene un valor en la asignación del 3-SAT, mientras que los números $x_j$ y $y_j$ no tienen 1 en esas
    posiciones, por tanto no influyen en la suma. Luego las $n$ primeras columnas suman 1. 
    
    Ahora notemos que en cada cláusula $c_j$ debe haber al menos una variable en \texttt{True} para satisfacer la expresión, 
    esto en nuestra tabla se traduce en que, sin contar los número $x_j$ y $y_j$, cada una de las $j$ columnas con $n+1 \leq j \leq n+m$
    suma al menos 1, ya que cada $t_i$ aporta un 1 a la columna $j$ si aparece $x_i$ en $c_j$, y cada $j_i$ si aparece $\neg x_i$,
    además recordemos que solo tomamos $t_i$ o $f_i$ de forma tal que $x_i$ o $\neg x_i$ sean \texttt{True}. Además es fácil ver 
    que dicha suma es a lo sumo 3, ya que solo hay 3 variables en cada cláusula. Por tanto para cada columna $j$ hay 3 casos:

    \begin{itemize}
        \item Entre las primeras $n$ filas hay acumulada una suma de 1 en la columna $j$: Esto significa que la cláusula
              $c_j$ tiene exactamente una variable \texttt{True}, por lo que se tomó el elemento $y_j$ que aporta 1 a la
              suma de la columna. También se cumple que en la cláusula $c_j$ hay a lo sumo 2 variables \texttt{True} (solo
              hay 1) por lo que tambien se tomó el elemento $x_j$ que aporta 1 a la suma de la columna. Luego la suma de la
              columna $j$ es $1 + 1 + 1 = 3$
        \item Entre las primeras $n$ filas hay acumulada una suma de 2 en la columna $j$: Esto significa que la cláusula
              $c_j$ tiene exactamente dos variable \texttt{True}, por lo que no se tomó el elemento $y_j$ pero si el elemento $x_j$,
              ya que se cumple que hay a lo sumo 2 variables \texttt{True}. Por tanto la suma de la columna $j$ es $2 + 1 = 3$
        \item Entre las primeras $n$ filas hay acumulada una suma de 3 en la columna $j$: Esto significa que la cláusula
              $c_j$ tiene exactamente tres variable \texttt{True}, por lo que no se tomarán ni los elementos $y_i$, ni $x_i$.
              Por tanto la suma de la columna $j$ se mantiene en 3.
    \end{itemize}

    Nótese que las únicas variables que pueden influir en la suma de una columna $j$ son las que se han analizado
    en cada caso, y todas aportan a lo sumo solo 1 a la suma. Además todas estas operaciones (tomar números del conjunto
    y comprobar en qué cláusulas está cada variable) se pueden hacer en tiempo polinomial. Por tanto queda demostrado que 
    si existe una solución al problema de 3-SAT, podemos construir a partir de esta una solución al Subset sum problem en
    en tiempo polinomial. Vamos a hacer la demostración ahora en el otro sentido. Vamos a demostrar que si existe una 
    solución al Subset sum entonces podemos construir una solución para el 3-SAT.\\[10pt]

    Teniendo un subconjunto $C$ de números que suman $s$ vamos a hacer la asignación para el 3-SAT de la siguiente manera:
    Asignamos \texttt{True} a la variable $x_i$ si $t_i$ pertenece a $C$, mientras que asignamos \texttt{False} a la variable 
    $x_i$ si $f_i$ pertenece a $C$. Primeramente notemos que para un mismo valor de $i$ no pueden pertenecer simultáneamente 
    $t_i$ y $f_i$ a $C$, ya que sino alguno de los primeros $n$ dígitos sumaría 2, lo que contradice que $C$ sea una solución 
    para el Subset sum, esto significa que a cada variable $x_i$ se le está asignando un único valor. Ahora notemos que, incluso 
    si todos los $x_j$ y $y_j$ pertenecen al $C$, la suma de las $m$ columnas restantes será a lo sumo 2, por tanto para que $C$
    sea solución se tienen que haber escogido los $t_i$ y los $f_i$ de manera que cada uno aporte como mínimo 1 a la suma de
    cada una de estas columnas. Por la manera en que se han construido todos los números, esto se traduce a que en cada cláusula
    va a haber al menos una variable en \texttt{True} lo que hace que se satisfaga la ecuación, y por tanto la asignación
    realizada resuelve el 3-SAT. Comprobar los elementos de $C$ para asignar valores a los $x_i$ se pude hacer en tiempo lineal
    con un simple recorrido de $C$.

    Finalmente queda demostrado que se puede construir una solución para el 3-SAT a partir de una para el Subset sum en tiempo
    polinomial, por lo que como 3-SAT es NP hard entonces Subset Sum es también NP hard. Lo que a su vez implica que Subset Equality
    problem es NP hard. Finalmente como Subset Equality problem es NP, y ademas NP hard, es también NP completo.
    
\end{document}